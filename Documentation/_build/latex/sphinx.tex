% Generated by Sphinx.
\def\sphinxdocclass{report}
\documentclass[letterpaper,10pt,english]{sphinxmanual}
\usepackage[utf8]{inputenc}
\DeclareUnicodeCharacter{00A0}{\nobreakspace}
\usepackage{cmap}
\usepackage[T1]{fontenc}
\usepackage{babel}
\usepackage{times}
\usepackage[Bjarne]{fncychap}
\usepackage{longtable}
\usepackage{sphinx}
\usepackage{multirow}


\addto\captionsenglish{\renewcommand{\figurename}{Fig. }}
\addto\captionsenglish{\renewcommand{\tablename}{Table }}
\floatname{literal-block}{Listing }



\title{. Documentation}
\date{December 01, 2015}
\release{0.1.1}
\author{Author}
\newcommand{\sphinxlogo}{}
\renewcommand{\releasename}{Release}
\makeindex

\makeatletter
\def\PYG@reset{\let\PYG@it=\relax \let\PYG@bf=\relax%
    \let\PYG@ul=\relax \let\PYG@tc=\relax%
    \let\PYG@bc=\relax \let\PYG@ff=\relax}
\def\PYG@tok#1{\csname PYG@tok@#1\endcsname}
\def\PYG@toks#1+{\ifx\relax#1\empty\else%
    \PYG@tok{#1}\expandafter\PYG@toks\fi}
\def\PYG@do#1{\PYG@bc{\PYG@tc{\PYG@ul{%
    \PYG@it{\PYG@bf{\PYG@ff{#1}}}}}}}
\def\PYG#1#2{\PYG@reset\PYG@toks#1+\relax+\PYG@do{#2}}

\expandafter\def\csname PYG@tok@gd\endcsname{\def\PYG@tc##1{\textcolor[rgb]{0.63,0.00,0.00}{##1}}}
\expandafter\def\csname PYG@tok@gu\endcsname{\let\PYG@bf=\textbf\def\PYG@tc##1{\textcolor[rgb]{0.50,0.00,0.50}{##1}}}
\expandafter\def\csname PYG@tok@gt\endcsname{\def\PYG@tc##1{\textcolor[rgb]{0.00,0.27,0.87}{##1}}}
\expandafter\def\csname PYG@tok@gs\endcsname{\let\PYG@bf=\textbf}
\expandafter\def\csname PYG@tok@gr\endcsname{\def\PYG@tc##1{\textcolor[rgb]{1.00,0.00,0.00}{##1}}}
\expandafter\def\csname PYG@tok@cm\endcsname{\let\PYG@it=\textit\def\PYG@tc##1{\textcolor[rgb]{0.25,0.50,0.56}{##1}}}
\expandafter\def\csname PYG@tok@vg\endcsname{\def\PYG@tc##1{\textcolor[rgb]{0.73,0.38,0.84}{##1}}}
\expandafter\def\csname PYG@tok@m\endcsname{\def\PYG@tc##1{\textcolor[rgb]{0.13,0.50,0.31}{##1}}}
\expandafter\def\csname PYG@tok@mh\endcsname{\def\PYG@tc##1{\textcolor[rgb]{0.13,0.50,0.31}{##1}}}
\expandafter\def\csname PYG@tok@cs\endcsname{\def\PYG@tc##1{\textcolor[rgb]{0.25,0.50,0.56}{##1}}\def\PYG@bc##1{\setlength{\fboxsep}{0pt}\colorbox[rgb]{1.00,0.94,0.94}{\strut ##1}}}
\expandafter\def\csname PYG@tok@ge\endcsname{\let\PYG@it=\textit}
\expandafter\def\csname PYG@tok@vc\endcsname{\def\PYG@tc##1{\textcolor[rgb]{0.73,0.38,0.84}{##1}}}
\expandafter\def\csname PYG@tok@il\endcsname{\def\PYG@tc##1{\textcolor[rgb]{0.13,0.50,0.31}{##1}}}
\expandafter\def\csname PYG@tok@go\endcsname{\def\PYG@tc##1{\textcolor[rgb]{0.20,0.20,0.20}{##1}}}
\expandafter\def\csname PYG@tok@cp\endcsname{\def\PYG@tc##1{\textcolor[rgb]{0.00,0.44,0.13}{##1}}}
\expandafter\def\csname PYG@tok@gi\endcsname{\def\PYG@tc##1{\textcolor[rgb]{0.00,0.63,0.00}{##1}}}
\expandafter\def\csname PYG@tok@gh\endcsname{\let\PYG@bf=\textbf\def\PYG@tc##1{\textcolor[rgb]{0.00,0.00,0.50}{##1}}}
\expandafter\def\csname PYG@tok@ni\endcsname{\let\PYG@bf=\textbf\def\PYG@tc##1{\textcolor[rgb]{0.84,0.33,0.22}{##1}}}
\expandafter\def\csname PYG@tok@nl\endcsname{\let\PYG@bf=\textbf\def\PYG@tc##1{\textcolor[rgb]{0.00,0.13,0.44}{##1}}}
\expandafter\def\csname PYG@tok@nn\endcsname{\let\PYG@bf=\textbf\def\PYG@tc##1{\textcolor[rgb]{0.05,0.52,0.71}{##1}}}
\expandafter\def\csname PYG@tok@no\endcsname{\def\PYG@tc##1{\textcolor[rgb]{0.38,0.68,0.84}{##1}}}
\expandafter\def\csname PYG@tok@na\endcsname{\def\PYG@tc##1{\textcolor[rgb]{0.25,0.44,0.63}{##1}}}
\expandafter\def\csname PYG@tok@nb\endcsname{\def\PYG@tc##1{\textcolor[rgb]{0.00,0.44,0.13}{##1}}}
\expandafter\def\csname PYG@tok@nc\endcsname{\let\PYG@bf=\textbf\def\PYG@tc##1{\textcolor[rgb]{0.05,0.52,0.71}{##1}}}
\expandafter\def\csname PYG@tok@nd\endcsname{\let\PYG@bf=\textbf\def\PYG@tc##1{\textcolor[rgb]{0.33,0.33,0.33}{##1}}}
\expandafter\def\csname PYG@tok@ne\endcsname{\def\PYG@tc##1{\textcolor[rgb]{0.00,0.44,0.13}{##1}}}
\expandafter\def\csname PYG@tok@nf\endcsname{\def\PYG@tc##1{\textcolor[rgb]{0.02,0.16,0.49}{##1}}}
\expandafter\def\csname PYG@tok@si\endcsname{\let\PYG@it=\textit\def\PYG@tc##1{\textcolor[rgb]{0.44,0.63,0.82}{##1}}}
\expandafter\def\csname PYG@tok@s2\endcsname{\def\PYG@tc##1{\textcolor[rgb]{0.25,0.44,0.63}{##1}}}
\expandafter\def\csname PYG@tok@vi\endcsname{\def\PYG@tc##1{\textcolor[rgb]{0.73,0.38,0.84}{##1}}}
\expandafter\def\csname PYG@tok@nt\endcsname{\let\PYG@bf=\textbf\def\PYG@tc##1{\textcolor[rgb]{0.02,0.16,0.45}{##1}}}
\expandafter\def\csname PYG@tok@nv\endcsname{\def\PYG@tc##1{\textcolor[rgb]{0.73,0.38,0.84}{##1}}}
\expandafter\def\csname PYG@tok@s1\endcsname{\def\PYG@tc##1{\textcolor[rgb]{0.25,0.44,0.63}{##1}}}
\expandafter\def\csname PYG@tok@gp\endcsname{\let\PYG@bf=\textbf\def\PYG@tc##1{\textcolor[rgb]{0.78,0.36,0.04}{##1}}}
\expandafter\def\csname PYG@tok@sh\endcsname{\def\PYG@tc##1{\textcolor[rgb]{0.25,0.44,0.63}{##1}}}
\expandafter\def\csname PYG@tok@ow\endcsname{\let\PYG@bf=\textbf\def\PYG@tc##1{\textcolor[rgb]{0.00,0.44,0.13}{##1}}}
\expandafter\def\csname PYG@tok@sx\endcsname{\def\PYG@tc##1{\textcolor[rgb]{0.78,0.36,0.04}{##1}}}
\expandafter\def\csname PYG@tok@bp\endcsname{\def\PYG@tc##1{\textcolor[rgb]{0.00,0.44,0.13}{##1}}}
\expandafter\def\csname PYG@tok@c1\endcsname{\let\PYG@it=\textit\def\PYG@tc##1{\textcolor[rgb]{0.25,0.50,0.56}{##1}}}
\expandafter\def\csname PYG@tok@kc\endcsname{\let\PYG@bf=\textbf\def\PYG@tc##1{\textcolor[rgb]{0.00,0.44,0.13}{##1}}}
\expandafter\def\csname PYG@tok@c\endcsname{\let\PYG@it=\textit\def\PYG@tc##1{\textcolor[rgb]{0.25,0.50,0.56}{##1}}}
\expandafter\def\csname PYG@tok@mf\endcsname{\def\PYG@tc##1{\textcolor[rgb]{0.13,0.50,0.31}{##1}}}
\expandafter\def\csname PYG@tok@err\endcsname{\def\PYG@bc##1{\setlength{\fboxsep}{0pt}\fcolorbox[rgb]{1.00,0.00,0.00}{1,1,1}{\strut ##1}}}
\expandafter\def\csname PYG@tok@mb\endcsname{\def\PYG@tc##1{\textcolor[rgb]{0.13,0.50,0.31}{##1}}}
\expandafter\def\csname PYG@tok@ss\endcsname{\def\PYG@tc##1{\textcolor[rgb]{0.32,0.47,0.09}{##1}}}
\expandafter\def\csname PYG@tok@sr\endcsname{\def\PYG@tc##1{\textcolor[rgb]{0.14,0.33,0.53}{##1}}}
\expandafter\def\csname PYG@tok@mo\endcsname{\def\PYG@tc##1{\textcolor[rgb]{0.13,0.50,0.31}{##1}}}
\expandafter\def\csname PYG@tok@kd\endcsname{\let\PYG@bf=\textbf\def\PYG@tc##1{\textcolor[rgb]{0.00,0.44,0.13}{##1}}}
\expandafter\def\csname PYG@tok@mi\endcsname{\def\PYG@tc##1{\textcolor[rgb]{0.13,0.50,0.31}{##1}}}
\expandafter\def\csname PYG@tok@kn\endcsname{\let\PYG@bf=\textbf\def\PYG@tc##1{\textcolor[rgb]{0.00,0.44,0.13}{##1}}}
\expandafter\def\csname PYG@tok@o\endcsname{\def\PYG@tc##1{\textcolor[rgb]{0.40,0.40,0.40}{##1}}}
\expandafter\def\csname PYG@tok@kr\endcsname{\let\PYG@bf=\textbf\def\PYG@tc##1{\textcolor[rgb]{0.00,0.44,0.13}{##1}}}
\expandafter\def\csname PYG@tok@s\endcsname{\def\PYG@tc##1{\textcolor[rgb]{0.25,0.44,0.63}{##1}}}
\expandafter\def\csname PYG@tok@kp\endcsname{\def\PYG@tc##1{\textcolor[rgb]{0.00,0.44,0.13}{##1}}}
\expandafter\def\csname PYG@tok@w\endcsname{\def\PYG@tc##1{\textcolor[rgb]{0.73,0.73,0.73}{##1}}}
\expandafter\def\csname PYG@tok@kt\endcsname{\def\PYG@tc##1{\textcolor[rgb]{0.56,0.13,0.00}{##1}}}
\expandafter\def\csname PYG@tok@sc\endcsname{\def\PYG@tc##1{\textcolor[rgb]{0.25,0.44,0.63}{##1}}}
\expandafter\def\csname PYG@tok@sb\endcsname{\def\PYG@tc##1{\textcolor[rgb]{0.25,0.44,0.63}{##1}}}
\expandafter\def\csname PYG@tok@k\endcsname{\let\PYG@bf=\textbf\def\PYG@tc##1{\textcolor[rgb]{0.00,0.44,0.13}{##1}}}
\expandafter\def\csname PYG@tok@se\endcsname{\let\PYG@bf=\textbf\def\PYG@tc##1{\textcolor[rgb]{0.25,0.44,0.63}{##1}}}
\expandafter\def\csname PYG@tok@sd\endcsname{\let\PYG@it=\textit\def\PYG@tc##1{\textcolor[rgb]{0.25,0.44,0.63}{##1}}}

\def\PYGZbs{\char`\\}
\def\PYGZus{\char`\_}
\def\PYGZob{\char`\{}
\def\PYGZcb{\char`\}}
\def\PYGZca{\char`\^}
\def\PYGZam{\char`\&}
\def\PYGZlt{\char`\<}
\def\PYGZgt{\char`\>}
\def\PYGZsh{\char`\#}
\def\PYGZpc{\char`\%}
\def\PYGZdl{\char`\$}
\def\PYGZhy{\char`\-}
\def\PYGZsq{\char`\'}
\def\PYGZdq{\char`\"}
\def\PYGZti{\char`\~}
% for compatibility with earlier versions
\def\PYGZat{@}
\def\PYGZlb{[}
\def\PYGZrb{]}
\makeatother

\renewcommand\PYGZsq{\textquotesingle}

\begin{document}

\maketitle
\tableofcontents
\phantomsection\label{index::doc}


Contents:


\chapter{Branch3D module}
\label{Branch3D:branch3d-module}\label{Branch3D::doc}\label{Branch3D:welcome-to-s-documentation}\label{Branch3D:module-Branch3D}\index{Branch3D (module)}
This module contains the Branch class (one branch of the tree)  and the Nodes class
\index{Branch (class in Branch3D)}

\begin{fulllineitems}
\phantomsection\label{Branch3D:Branch3D.Branch}\pysiglinewithargsret{\strong{class }\code{Branch3D.}\bfcode{Branch}}{\emph{mesh}, \emph{init\_node}, \emph{init\_dir}, \emph{init\_tri}, \emph{l}, \emph{angle}, \emph{w}, \emph{nodes}, \emph{brother\_nodes}, \emph{Nsegments}}{}
Class that contains a branch of the fractal tree
\begin{quote}\begin{description}
\item[{Parameters}] \leavevmode\begin{itemize}
\item {} 
\textbf{\texttt{mesh}} -- an object of the mesh class, where the fractal tree will grow

\item {} 
\textbf{\texttt{init\_node}} (\emph{\texttt{int}}) -- initial node to grow the branch. This is an index that refers to a node in the nodes.nodes array.

\item {} 
\textbf{\texttt{init\_dir}} (\emph{\texttt{array}}) -- initial direction to grow the branch. In general, it refers to the direction of the last segment of the mother brach.

\item {} 
\textbf{\texttt{init\_tri}} (\emph{\texttt{int}}) -- the index of triangle of the mesh where the init\_node sits.

\item {} 
\textbf{\texttt{l}} (\emph{\texttt{float}}) -- total length of the branch

\item {} 
\textbf{\texttt{angle}} (\emph{\texttt{float}}) -- angle (rad) with respect to the init\_dir in the plane of the init\_tri triangle

\item {} 
\textbf{\texttt{w}} (\emph{\texttt{float}}) -- repulsitivity parameter. Controls how much the branches repel each other.

\item {} 
\textbf{\texttt{nodes}} -- the object of the class nodes that contains all the nodes of the existing branches.

\item {} 
\textbf{\texttt{brother\_nodes}} (\emph{\texttt{list}}) -- the nodes of the brother and mother branches, to be excluded from the collision detection between branches.

\item {} 
\textbf{\texttt{Nsegments}} (\emph{\texttt{int}}) -- number of segments to divide the branch.

\end{itemize}

\end{description}\end{quote}
\index{child (Branch3D.Branch attribute)}

\begin{fulllineitems}
\phantomsection\label{Branch3D:Branch3D.Branch.child}\pysigline{\bfcode{child}}
\emph{list}

contains the indexes of the child branches. It is not assigned when created.

\end{fulllineitems}

\index{dir (Branch3D.Branch attribute)}

\begin{fulllineitems}
\phantomsection\label{Branch3D:Branch3D.Branch.dir}\pysigline{\bfcode{dir}}
\emph{array}

vector direction of the last segment of the branch.

\end{fulllineitems}

\index{nodes (Branch3D.Branch attribute)}

\begin{fulllineitems}
\phantomsection\label{Branch3D:Branch3D.Branch.nodes}\pysigline{\bfcode{nodes}}
\emph{list}

contains the node indices of the branch. The node coordinates can be retrieved using nodes.nodes{[}i{]}

\end{fulllineitems}

\index{triangles (Branch3D.Branch attribute)}

\begin{fulllineitems}
\phantomsection\label{Branch3D:Branch3D.Branch.triangles}\pysigline{\bfcode{triangles}}
\emph{list}

contains the indices of the triangles from the mesh where every node of the branch lies.

\end{fulllineitems}

\index{tri (Branch3D.Branch attribute)}

\begin{fulllineitems}
\phantomsection\label{Branch3D:Branch3D.Branch.tri}\pysigline{\bfcode{tri}}
\emph{int}

triangle index where last node sits.

\end{fulllineitems}

\index{growing (Branch3D.Branch attribute)}

\begin{fulllineitems}
\phantomsection\label{Branch3D:Branch3D.Branch.growing}\pysigline{\bfcode{growing}}
\emph{bool}

False if the branch collide or is out of the surface. True otherwise.

\end{fulllineitems}

\index{add\_node\_to\_queue() (Branch3D.Branch method)}

\begin{fulllineitems}
\phantomsection\label{Branch3D:Branch3D.Branch.add_node_to_queue}\pysiglinewithargsret{\bfcode{add\_node\_to\_queue}}{\emph{mesh}, \emph{init\_node}, \emph{dir}}{}
Functions that projects a node in the mesh surface and it to the queue is it lies in the surface.
\begin{quote}\begin{description}
\item[{Parameters}] \leavevmode\begin{itemize}
\item {} 
\textbf{\texttt{mesh}} -- an object of the mesh class, where the fractal tree will grow

\item {} 
\textbf{\texttt{init\_node}} (\emph{\texttt{array}}) -- vector that contains the coordinates of the last node added in the branch.

\item {} 
\textbf{\texttt{dir}} (\emph{\texttt{array}}) -- vector that contains the direction from the init\_node to the node to project.

\end{itemize}

\item[{Returns}] \leavevmode
\textbf{success} --
true if the new node is in the triangle.

\item[{Return type}] \leavevmode
bool

\end{description}\end{quote}

\end{fulllineitems}


\end{fulllineitems}

\index{Nodes (class in Branch3D)}

\begin{fulllineitems}
\phantomsection\label{Branch3D:Branch3D.Nodes}\pysiglinewithargsret{\strong{class }\code{Branch3D.}\bfcode{Nodes}}{\emph{init\_node}}{}
A class containing the nodes of the branches plus some fuctions to compute distance related quantities.
\begin{quote}\begin{description}
\item[{Parameters}] \leavevmode
\textbf{\texttt{init\_node}} (\emph{\texttt{array}}) -- an array with the coordinates of the initial node of the first branch.

\end{description}\end{quote}
\index{nodes (Branch3D.Nodes attribute)}

\begin{fulllineitems}
\phantomsection\label{Branch3D:Branch3D.Nodes.nodes}\pysigline{\bfcode{nodes}}
\emph{list}

list of arrays containing the coordinates of the nodes

\end{fulllineitems}

\index{last\_node (Branch3D.Nodes attribute)}

\begin{fulllineitems}
\phantomsection\label{Branch3D:Branch3D.Nodes.last_node}\pysigline{\bfcode{last\_node}}
\emph{int}

last added node.

\end{fulllineitems}

\index{end\_nodes (Branch3D.Nodes attribute)}

\begin{fulllineitems}
\phantomsection\label{Branch3D:Branch3D.Nodes.end_nodes}\pysigline{\bfcode{end\_nodes}}
\emph{list}

a list containing the indices of all end nodes (nodes that are not connected) of the tree.

\end{fulllineitems}

\index{tree (Branch3D.Nodes attribute)}

\begin{fulllineitems}
\phantomsection\label{Branch3D:Branch3D.Nodes.tree}\pysigline{\bfcode{tree}}
\emph{scipy.spatial.cKDTree}

a k-d tree to compute the distance from any point to the closest node in the tree. It is updated once a branch is finished.

\end{fulllineitems}

\index{collision\_tree (Branch3D.Nodes attribute)}

\begin{fulllineitems}
\phantomsection\label{Branch3D:Branch3D.Nodes.collision_tree}\pysigline{\bfcode{collision\_tree}}
\emph{scipy.spatial.cKDTree}

a k-d tree to compute the distance from any point to the closest node in the tree, except from the brother and mother branches. It is used to check collision between branches.

\end{fulllineitems}

\index{add\_nodes() (Branch3D.Nodes method)}

\begin{fulllineitems}
\phantomsection\label{Branch3D:Branch3D.Nodes.add_nodes}\pysiglinewithargsret{\bfcode{add\_nodes}}{\emph{queue}}{}
This function stores a list of nodes of a branch and returns the node indices. It also updates the tree to compute distances.
\begin{quote}\begin{description}
\item[{Parameters}] \leavevmode
\textbf{\texttt{queue}} (\emph{\texttt{list}}) -- a list of arrays containing the coordinates of the nodes of one branch.

\item[{Returns}] \leavevmode
\textbf{nodes\_id} --
the indices of the added nodes.

\item[{Return type}] \leavevmode
list

\end{description}\end{quote}

\end{fulllineitems}

\index{collision() (Branch3D.Nodes method)}

\begin{fulllineitems}
\phantomsection\label{Branch3D:Branch3D.Nodes.collision}\pysiglinewithargsret{\bfcode{collision}}{\emph{point}}{}
This function returns the distance between one point and the closest node in the tree and the index of the closest node using the collision\_tree.
\begin{quote}\begin{description}
\item[{Parameters}] \leavevmode
\textbf{\texttt{point}} (\emph{\texttt{array}}) -- the coordinates of the point to calculate the distance from.

\item[{Returns}] \leavevmode
\textbf{collision} --
(distance to the closest node, index of the closest node)

\item[{Return type}] \leavevmode
tuple

\end{description}\end{quote}

\end{fulllineitems}

\index{distance\_from\_node() (Branch3D.Nodes method)}

\begin{fulllineitems}
\phantomsection\label{Branch3D:Branch3D.Nodes.distance_from_node}\pysiglinewithargsret{\bfcode{distance\_from\_node}}{\emph{node}}{}
This function returns the distance from any node to the closest node in the tree.
\begin{quote}\begin{description}
\item[{Parameters}] \leavevmode
\textbf{\texttt{node}} (\emph{\texttt{int}}) -- the index of the node to calculate the distance from.

\item[{Returns}] \leavevmode
\textbf{d} --
the distance between specified node and the closest node in the tree.

\item[{Return type}] \leavevmode
float

\end{description}\end{quote}

\end{fulllineitems}

\index{distance\_from\_point() (Branch3D.Nodes method)}

\begin{fulllineitems}
\phantomsection\label{Branch3D:Branch3D.Nodes.distance_from_point}\pysiglinewithargsret{\bfcode{distance\_from\_point}}{\emph{point}}{}
This function returns the distance from any point to the closest node in the tree.
\begin{quote}\begin{description}
\item[{Parameters}] \leavevmode
\textbf{\texttt{point}} (\emph{\texttt{array}}) -- the coordinates of the point to calculate the distance from.

\item[{Returns}] \leavevmode
\textbf{d} --
the distance between point and the closest node in the tree.

\item[{Return type}] \leavevmode
float

\end{description}\end{quote}

\end{fulllineitems}

\index{gradient() (Branch3D.Nodes method)}

\begin{fulllineitems}
\phantomsection\label{Branch3D:Branch3D.Nodes.gradient}\pysiglinewithargsret{\bfcode{gradient}}{\emph{point}}{}
This function returns the gradient of the distance from the existing points of the tree from any point. It uses a central finite difference approximation.
\begin{quote}\begin{description}
\item[{Parameters}] \leavevmode
\textbf{\texttt{point}} (\emph{\texttt{array}}) -- the coordinates of the point to calculate the gradient of the distance from.

\item[{Returns}] \leavevmode
\textbf{grad} --
(x,y,z) components of gradient of the distance.

\item[{Return type}] \leavevmode
array

\end{description}\end{quote}

\end{fulllineitems}

\index{update\_collision\_tree() (Branch3D.Nodes method)}

\begin{fulllineitems}
\phantomsection\label{Branch3D:Branch3D.Nodes.update_collision_tree}\pysiglinewithargsret{\bfcode{update\_collision\_tree}}{\emph{nodes\_to\_exclude}}{}
This function updates the collision\_tree excluding a list of nodes from all the nodes in the tree. If all the existing nodes are excluded, one distant node is added.
\begin{quote}\begin{description}
\item[{Parameters}] \leavevmode
\textbf{\texttt{nodes\_to\_exclude}} (\emph{\texttt{list}}) -- contains the nodes to exclude from the tree. Usually it should be the mother and the brother branch nodes.

\item[{Returns}] \leavevmode
none

\end{description}\end{quote}

\end{fulllineitems}


\end{fulllineitems}



\chapter{Fractal\_Tree\_3D module}
\label{Fractal_Tree_3D:fractal-tree-3d-module}\label{Fractal_Tree_3D::doc}

\chapter{mesh module}
\label{mesh:mesh-module}\label{mesh::doc}\label{mesh:module-mesh}\index{mesh (module)}
This module contains the mesh class. This class is the triangular surface where the fractal tree is grown.
\index{Mesh (class in mesh)}

\begin{fulllineitems}
\phantomsection\label{mesh:mesh.Mesh}\pysiglinewithargsret{\strong{class }\code{mesh.}\bfcode{Mesh}}{\emph{filename}}{}
Class that contains the mesh where fractal tree is grown. It must be Wavefront .obj file. Be careful on how the normals are defined. It can change where an specified angle will go.
\begin{quote}\begin{description}
\item[{Parameters}] \leavevmode
\textbf{\texttt{filename}} (\emph{\texttt{str}}) -- the path and filename of the .obj file.

\end{description}\end{quote}
\index{verts (mesh.Mesh attribute)}

\begin{fulllineitems}
\phantomsection\label{mesh:mesh.Mesh.verts}\pysigline{\bfcode{verts}}
\emph{array}

a numpy array that contains all the nodes of the mesh. verts{[}i,j{]}, where i is the node index and j={[}0,1,2{]} is the coordinate (x,y,z).

\end{fulllineitems}

\index{connectivity (mesh.Mesh attribute)}

\begin{fulllineitems}
\phantomsection\label{mesh:mesh.Mesh.connectivity}\pysigline{\bfcode{connectivity}}
\emph{array}

a numpy array that contains all the connectivity of the triangles of the mesh. connectivity{[}i,j{]}, where i is the triangle index and j={[}0,1,2{]} is node index.

\end{fulllineitems}

\index{normals (mesh.Mesh attribute)}

\begin{fulllineitems}
\phantomsection\label{mesh:mesh.Mesh.normals}\pysigline{\bfcode{normals}}
\emph{array}

a numpy array that contains all the normals of the triangles of the mesh. normals{[}i,j{]}, where i is the triangle index and j={[}0,1,2{]} is normal coordinate (x,y,z).

\end{fulllineitems}

\index{node\_to\_tri (mesh.Mesh attribute)}

\begin{fulllineitems}
\phantomsection\label{mesh:mesh.Mesh.node_to_tri}\pysigline{\bfcode{node\_to\_tri}}
\emph{dict}

a dictionary that relates a node to the triangles that it is connected. It is the inverse relation of connectivity. The triangles are stored as a list for each node.

\end{fulllineitems}

\index{tree (mesh.Mesh attribute)}

\begin{fulllineitems}
\phantomsection\label{mesh:mesh.Mesh.tree}\pysigline{\bfcode{tree}}
\emph{scipy.spatial.cKDTree}

a k-d tree to compute the distance from any point to the closest node in the mesh.

\end{fulllineitems}

\index{loadOBJ() (mesh.Mesh method)}

\begin{fulllineitems}
\phantomsection\label{mesh:mesh.Mesh.loadOBJ}\pysiglinewithargsret{\bfcode{loadOBJ}}{\emph{filename}}{}
This function reads a .obj mesh file
\begin{quote}\begin{description}
\item[{Parameters}] \leavevmode
\textbf{\texttt{filename}} (\emph{\texttt{str}}) -- the path and filename of the .obj file.

\item[{Returns}] \leavevmode
\textbf{verts} --
a numpy array that contains all the nodes of the mesh. verts{[}i,j{]}, where i is the node index and j={[}0,1,2{]} is the coordinate (x,y,z).
connectivity (array): a numpy array that contains all the connectivity of the triangles of the mesh. connectivity{[}i,j{]}, where i is the triangle index and j={[}0,1,2{]} is node index.

\item[{Return type}] \leavevmode
array

\end{description}\end{quote}

\end{fulllineitems}

\index{project\_new\_point() (mesh.Mesh method)}

\begin{fulllineitems}
\phantomsection\label{mesh:mesh.Mesh.project_new_point}\pysiglinewithargsret{\bfcode{project\_new\_point}}{\emph{point}}{}
This function projects any point to the surface defined by the mesh.
\begin{quote}\begin{description}
\item[{Parameters}] \leavevmode
\textbf{\texttt{point}} (\emph{\texttt{array}}) -- coordinates of the point to project.

\item[{Returns}] \leavevmode
\textbf{projected\_point} --
the coordinates of the projected point that lies in the surface.
intriangle (int): the index of the triangle where the projected point lies. If the point is outside surface, intriangle=-1.

\item[{Return type}] \leavevmode
array

\end{description}\end{quote}

\end{fulllineitems}


\end{fulllineitems}



\chapter{sphere\_parameters module}
\label{sphere_parameters::doc}\label{sphere_parameters:module-sphere_parameters}\label{sphere_parameters:sphere-parameters-module}\index{sphere\_parameters (module)}
Created on Tue Nov 24 10:33:00 2015

@author: fsc
\index{Parameters (class in sphere\_parameters)}

\begin{fulllineitems}
\phantomsection\label{sphere_parameters:sphere_parameters.Parameters}\pysigline{\strong{class }\code{sphere\_parameters.}\bfcode{Parameters}}
\end{fulllineitems}



\chapter{Indices and tables}
\label{index:indices-and-tables}\begin{itemize}
\item {} 
\DUspan{xref,std,std-ref}{genindex}

\item {} 
\DUspan{xref,std,std-ref}{modindex}

\item {} 
\DUspan{xref,std,std-ref}{search}

\end{itemize}


\renewcommand{\indexname}{Python Module Index}
\begin{theindex}
\def\bigletter#1{{\Large\sffamily#1}\nopagebreak\vspace{1mm}}
\bigletter{b}
\item {\texttt{Branch3D}}, \pageref{Branch3D:module-Branch3D}
\indexspace
\bigletter{m}
\item {\texttt{mesh}}, \pageref{mesh:module-mesh}
\indexspace
\bigletter{s}
\item {\texttt{sphere\_parameters}}, \pageref{sphere_parameters:module-sphere_parameters}
\end{theindex}

\renewcommand{\indexname}{Index}
\printindex
\end{document}
